\chapter{Anexo}
\label{cap:anexo}

\section{Ecuaciones lógicas estación distribución}
\label{sec:anexo_distribucion}

\subsubsection{Ecuaciones de transición de etapas}

\begin{table}[H]
\begin{center}

\renewcommand{\arraystretch}{1.5}
\begin{tabular}{|M{1.5cm}|M{6cm}|M{6cm}|}
\hline
\textbf{Etapa} & 
\textbf{Set} & 
\textbf{Reset} \\
\hline
E0  &  FSM + Iniciar\_programa & E1 + Parada\_emergencia \\
\hline
E1  &  Inicio\_HMI * (E0 + (t\_inicio  * (E17 + E20+ E22 + E23)) & E2 + E4 + Parada\_emergencia  \\
\hline
E2  &  E1 * Inicio\_cinta * Cargador\_HMI & E3 + Parada\_emergencia \\
\hline
E3  &  E2 * Tiempo\_separador & E7 + Parada\_emergencia \\
\hline
E4  &  E1 * Pieza\_cragador * $\overline{\text{Cargador\_HMI}}$ & E5 + Parada\_emergencia \\
\hline
E5  &  E4 + (Corredera\_extendida * Medio\_cinta) & E6 + Parada\_emergencia \\
\hline
E6  &  E5 * Corredera\_retraida & E7 + Parada\_emergencia \\
\hline
E7  &  (E3 + E6) * Identificador\_piezas & E8 + Parada\_emergencia \\
\hline
E8  &  E7 * Tiempo\_sensor\_piezas & E9 + E12 + Parada\_emergencia \\
\hline
E9 & 
E8 * espera\_sensor\_piezas * Contador\_pieza * 
( (Pieza\_metalica\_HMI * Pieza\_metalica * Pieza\_rosa) + 
(Pieza\_rosa\_HMI * $\overline{\text{Pieza\_metalica}}$ * Pieza\_rosa) + 
(Pieza\_negra\_HMI * $\overline{\text{Pieza\_metalica}}$ * $\overline{\text{Pieza\_rosa}}$) )
 & E10 + Parada\_emergencia \\
\hline
E10  &  E9 * $\overline{\text{Final\_cinta}}$ & E11 + Parada\_emergencia \\
\hline
E11  &  E10 * Comunicacion\_PLC\_1.Recive\_fase\_1 & E14 + E18 + Parada\_emergencia \\
\hline

\end{tabular}

\caption{Ecuaciones de transición de estados de la estación distribución.}
\label{cuadro:transiciones_distribucion_1}
\end{center}
\end{table}

\begin{table}[H]
\begin{center}

\renewcommand{\arraystretch}{1.5}
\begin{tabular}{|M{1.5cm}|M{6cm}|M{6cm}|}
\hline
\textbf{Etapa} & 
\textbf{Set} & 
\textbf{Reset} \\
\hline
E12  &  E8 * $\overline{\text{E9}}$ * espera\_sensor\_piezas\_2 & E13 + Parada\_emergencia \\
\hline
E13  &  E12 * tiempo\_separador\_ret & E23 + Parada\_emergencia \\
\hline
E14  &  E11 *  Comunicacion\_PLC\_1.Pieza\_mal\_orientada & E15 + Parada\_emergencia \\
\hline
E15  &  E14 * $\overline{\text{Final\_cinta}}$ & E16 + Parada\_emergencia \\
\hline
E16  &  E15 * T\_sep\_2 & E17 + Parada\_emergencia \\
\hline
E17  &  E16 * Inicio\_cinta & E1 + Parada\_emergencia \\
\hline
E18  &  E11 *  Comunicacion\_PLC\_1.Secuencia\_acabada & E19 + Parada\_emergencia \\
\hline
E19  &  E18 * $\overline{\text{UR\_IN.Bits.Register[0]}}$  * t\_espera & E20 + E21 + Parada\_emergencia \\
\hline
E20  &  E19 * UR\_IN.Bits.Register[0] & E1 + Parada\_emergencia \\
\hline
E21  &  E19 * UR\_IN.Bits.Register[1] & E22 + Parada\_emergencia \\
\hline
E22  &  E21 * UR\_IN.Bits.Register[2] & E1 + Parada\_emergencia \\
\hline
E23  &  E13 * Inicio\_cinta & E1 + Parada\_emergencia \\
\hline

\end{tabular}

\caption{Ecuaciones de transición de estados de la estación distribución.}
\label{cuadro:transiciones_distribucion_2}
\end{center}
\end{table}

\subsubsection{Ecuaciones de ejecución de acciones}

\begin{table}[H]
\begin{center}

\renewcommand{\arraystretch}{1.5}
\begin{tabular}{|M{5cm}|M{8.5cm}|}
\hline
\textbf{Acción} & 
\textbf{Lógica de activación} \\ 
\hline
Avance\_cinta &  E2 + E3 + E6 + E7 + E9 + E10 \\
\hline
Retroceso\_cinta &  E12 + E13 * E14 + E15 + E16 \\
\hline
Separador &  E2 + E13 + E16 \\
\hline
Avance\_corredera &  E4 \\
\hline

\end{tabular}

\caption{Ecuaciones lógicas de las acciones de la estación distribución.}
\label{cuadro:acciones_distribucion}
\end{center}
\end{table}

\section{Ecuaciones lógicas estación unión}
\label{sec:anexo_union}

\subsubsection{Ecuaciones de transición de etapas}

\begin{table}[H]
\begin{center}

\renewcommand{\arraystretch}{1.5}
\begin{tabular}{|M{1.5cm}|M{6cm}|M{6cm}|}
\hline
\textbf{Etapa} & 
\textbf{Set} & 
\textbf{Reset} \\
\hline
E0  &  FSM + Iniciar\_programa\_2 & E1 + Parada\_emergencia\_2  \\
\hline
E1  &  E0 + (Comunicacion\_PLC\_2.Secuencia\_ACK * E15) + (Comunicacion\_PLC\_2.Pieza\_mal\_ACK * E18) & E2 + Parada\_emergencia\_2  \\
\hline
E2  &  E1 * Comunicacion\_PLC\_2.Recive\_fase\_1 & E3 + Parada\_emergencia\_2  \\
\hline
E3  &  E2 * (Inicio\_cinta\_1 + T\_pieza\_negra) & E4 + Parada\_emergencia\_2  \\
\hline
E4  &  E3 * ((Medio\_cinta\_1 * T\_espera) + (T\_pieza\_negra\_2)) & (E5 * E13) + E16 +  Parada\_emergencia\_2  \\
\hline
E5  &  E4 * T\_orientacion\_2 ( (Comunicacion\_PLC\_2.Tipo\_de\_pieza $\leq$ 2 * Orientacion\_correcta) + (Comunicacion\_PLC\_2.Tipo\_de\_pieza == 3 * $\overline{\text{Orientacion\_correcta}}$) ) & E6 + Parada\_emergencia\_2  \\
\hline
E6  &  E5 * T\_derivador & E7 + Parada\_emergencia\_2  \\
\hline
E7  &  E6 * E14 * Carro\_extendido * T\_E7 & E8 + Parada\_emergencia\_2  \\
\hline
E8  &  E7 * $\overline{\text{Ventosa\_arriba}}$ & E9 + Parada\_emergencia\_2  \\
\hline
E9 & E8 * Pieza\_succionada & E10 + Parada\_emergencia\_2  \\
\hline
E10  &  E9 * Ventosa\_arriba & E11 + Parada\_emergencia\_2  \\
\hline
E11  &  E10 * (Carro\_retraido + T\_carro) & E12 + Parada\_emergencia\_2  \\
\hline
E12  &  E11 * $\overline{\text{Ventosa\_arriba}}$ & E17 + Parada\_emergencia\_2 \\
\hline
E13  &  E4 * ( (Comunicacion\_PLC\_2.Tipo\_de\_pieza $\leq$ 2 * Orientacion\_correcta) + (Comunicacion\_PLC\_2.Tipo\_de\_pieza == 3 * $\overline{\text{Orientacion\_correcta}}$) ) & E14 + Parada\_emergencia\_2 \\
\hline
\end{tabular}

\caption{Ecuaciones de transición de estados de la estación unión.}
\label{cuadro:transiciones_union_1}
\end{center}
\end{table}

\begin{table}[H]
\begin{center}

\renewcommand{\arraystretch}{1.5}
\begin{tabular}{|M{1.5cm}|M{6cm}|M{6cm}|}
\hline
\textbf{Etapa} & 
\textbf{Set} & 
\textbf{Reset} \\
\hline
E14  &  E13 *  Final\_cinta\_2 * T\_E14 & E7 + Parada\_emergencia\_2 \\
\hline
E15  &  E17 * $\overline{\text{Final\_cinta\_1}}$ & E1 + Parada\_emergencia\_2 \\
\hline
E16  &  E4 * $\overline{\text{E5}}$ * T\_orientacion * ( (Comunicacion\_PLC\_2.Tipo\_de\_pieza $\leq$ 2 * Orientacion\_correcta) + (Comunicacion\_PLC\_2.Tipo\_de\_pieza == 3 * $\overline{\text{Orientacion\_correcta}}$) ) & E18 + Parada\_emergencia\_2 \\
\hline
E17  &  E12 * T\_tapa & E15 + Parada\_emergencia\_2 \\
\hline
E18  &  E16 *  Boton\_HMI\_pieza & E1 + Parada\_emergencia\_2 \\
\hline

\end{tabular}

\caption{Ecuaciones de transición de estados de la estación unión.}
\label{cuadro:transiciones_union_2}
\end{center}
\end{table}

\subsubsection{Ecuaciones de ejecución de acciones}

\begin{table}[H]
\begin{center}

\renewcommand{\arraystretch}{1.5}
\begin{tabular}{|M{5cm}|M{8.5cm}|}
\hline
\textbf{Acción} & 
\textbf{Lógica de activación} \\ 
\hline
Avance\_cinta\_1 &  E2 + E3 + E5 + E17 \\
\hline
Avance\_cinta\_2 &  E13 \\
\hline
Retroceso\_cinta\_1 & E18 \\
\hline
Extender\_separador &  E5 + E6 + E7 + E8 + E9 + E10 + E11 \\
\hline
Retraer\_tope &  E5 \\
\hline
Avance\_carro &  E6 \\
\hline
Retroceso\_carro &  E10 \\
\hline
Ventosa\_abajo &  E7 + E8 + E10 \\
\hline
Vacío\_conectado & E8 + E9 + E10 + E11 \\
\hline


\end{tabular}

\caption{Ecuaciones lógicas de las acciones de la estación distribución.}
\label{cuadro:acciones_distribucion}
\end{center}
\end{table}


