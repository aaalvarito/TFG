\chapter{Objetivos}
\label{cap:capitulo2}

\begin{flushright}
\begin{minipage}[]{10cm}
\emph{Quizás algún fragmento de libro inspirador...}\\
\end{minipage}\\

Autor, \textit{Título}\\
\end{flushright}

\vspace{1cm}

Escribe aquí un párrafo explicando brevemente lo que vas a contar en este capítulo. En este capítulo lo ideal es explicar cuáles han sido los objetivos que te has fijado conseguir con tu trabajo, qué requisitos ha de respetar el resultado final, y cómo lo has llevado a cabo; esto es, cuál ha sido tu plan de trabajo.\\

\section{Descripción del problema}
\label{sec:descripcion}

Cuenta aquí el objetivo u objetivos generales y, a continuación, concrétalos mediante objetivos específicos.

\section{Requisitos}
\label{sec:requisitos}

Describe los requisitos que ha de cumplir tu trabajo.

\section{Competencias}
\label{sec:competencias}

Enumera las competencias generales adquiridas y empleadas. Es decir, las que consideres que has adquirido durante la realización de tu TFG (las podrás encontrar en la guía docente de la asignatura TFG), así como las que creas que has adquirido en las distintas asignaturas del grado (descritas en sus respectivas guías docentes) y que has empleado para llevarlo a cabo.
 
\section{Metodología}
\label{sec:metodologia}

Qué paradigma de desarrollo software has seguido para alcanzar tus objetivos.

\section{Plan de trabajo}
\label{sec:plantrabajo}

Qué agenda has seguido. Si has ido manteniendo reuniones semanales, cumplimentando objetivos parciales, si has ido afinando poco a poco un producto final completo, etc.
