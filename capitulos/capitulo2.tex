\chapter{Objetivos}
\label{cap:capitulo2}

\begin{flushright}
\begin{minipage}[]{10cm}
\emph{La automatización bien aplicada no es una amenaza, sino una oportunidad para rediseñar el trabajo, mejorar la productividad y humanizar la industria}\\
\end{minipage}\\

Peter F. Drucker, \textit{ Management: Tasks, Responsibilities, Practices}\\
\end{flushright}

\vspace{1cm}

Tras haber contextualizado en el capítulo anterior los fundamentos de la automatización industrial y su relación con la robótica colaborativa, en esta sección se abordará el problema específico que se pretende resolver con este Trabajo Fin de Grado. Se definirá con precisión el objetivo principal del proyecto, centrado en la automatización de una línea de producción didáctica, y se establecerán las bases sobre las que se desarrollará la solución propuesta. Esta descripción permitirá entender el enfoque adoptado y servirá como punto de partida para detallar los requisitos, las competencias aplicadas y la metodología empleada en el desarrollo del sistema.

\section{Descripción del problema}
\label{sec:descripcion}

El objetivo principal de este Trabajo Fin de Grado es automatizar una línea de producción robotizada utilizando estaciones didácticas de Festo, incorporando un robot colaborativo (UR5e) y estableciendo un sistema de comunicaciones eficiente entre los distintos elementos que componen la celda. Se pretende lograr una integración realista y funcional que sirva tanto como entorno de pruebas para prácticas docentes como demostrador de capacidades en automatización industrial y robótica colaborativa.

Este desarrollo se centra en mejorar la interacción entre los diferentes componentes del sistema (PLCs, HMIs y cobot), garantizando una comunicación fluida y segura, así como un control intuitivo y robusto de todo el proceso productivo.

\section{Requisitos}
\label{sec:requisitos}

Durante el desarrollo del proyecto, se han definido y alcanzado los siguientes requisitos técnicos y funcionales:

\begin{itemize}
	\item Automatización del funcionamiento de las estaciones didácticas (distribución y unión) mediante programación en TIA Portal.

	\item Diseño y configuración de una interfaz HMI para el control del sistema y visualización de estados.

	\item Integración del robot colaborativo UR5e en el flujo de trabajo automatizado, realizando tareas de paletizado.

	\item Creación de una red de comunicaciones PROFINET entre PLCs, HMI y cobot, asegurando el intercambio de datos en tiempo real.

	\item Implementación de lógica de control secuencial utilizando Grafcet y guía GEMMA como base metodológica.

	\item Montaje, conexionado y validación del sistema completo en condiciones reales de funcionamiento.
\end{itemize}

\section{Competencias}
\label{sec:competencias}

Las competencias desarrolladas y aplicadas en este proyecto están contenidas dentro del grado en Ingeniería Robótica Software, y en particular con las asignaturas de robótica industrial, redes de ordenadores, y la asignatura propia de trabajo de fin de grado. A continuación se enumeran las competencias adquiridas:

\begin{itemize}
	\item Se han aplicado los conocimientos adquiridos de forma profesional, resolviendo problemas y elaborando argumentos dentro del ámbito de la robótica. Se ha demostrado capacidad para comunicar información, ideas y soluciones a públicos especializados y no especializados, y se ha fomentado la autonomía en el aprendizaje. Todo ello ha culminado en la realización y defensa de un proyecto que integra y sintetiza las competencias del grado.

	\item Capacidad de diseñar y programar sistemas en red aplicando conceptos de arquitectura de red, protocolos e interfaces de comunicaciones.

	\item Capacidad de diseñar, planificar y programar sistemas de manipulación robóticos.

	\item Capacidad de diseñar robots y sistemas inteligentes atendiendo a los elementos de sensorización y actuación más adecuados dependiendo de la aplicación, los requerimientos del sistema y las condiciones del entorno.

\end{itemize}
 
\section{Metodología}
\label{sec:metodologia}

El desarrollo del presente trabajo se ha basado en una metodología estructurada y secuencial, orientada a la automatización industrial y fundamentada en los estándares Grafcet y Guía GEMMA, los cuales ya fueron explicados en el capítulo 1.

Para definir el comportamiento lógico del sistema automatizado, se utilizó el modelo GRAFCET, que permitió representar de forma clara y estructurada la secuencia de funcionamiento de cada estación, identificando las diferentes etapas del proceso, las transiciones entre ellas basadas en condiciones lógicas, y las acciones asociadas a cada estado. Después se procedió a su traducción al lenguaje Ladder para su implementación en los PLCs. De forma complementaria, se empleó la Guía GEMMA para estructurar los distintos modos de operación del sistema, incluyendo funcionamiento normal, parada, arranque y gestión de fallos, lo que permitió definir una lógica de control más robusta y segura, especialmente ante condiciones anómalas o situaciones de emergencia.

Respecto a la programación del cobot UR5e, se empleó una metodología centrada en la programación mediante la interfaz gráfica PolyScope y ayudándose de la herramienta de paletizado de esta. Una vez desarrollada la lógica, se trasladó directamente al entorno físico y se integró con las estaciones didácticas utilizando comunicación PROFINET.

\section{Plan de trabajo}
\label{sec:plantrabajo}

El desarrollo del proyecto comenzó con la instalación del primer PLC y la interfaz HMI, así como su correspondiente configuración en el entorno TIA Portal. A continuación, se procedió a la instalación de la estación de distribución, realizando el conexionado de todas las entradas y salidas al PLC para su control. Una vez completada esta etapa, se inició la programación de la lógica de control de dicha estación en el PLC y el diseño de la interfaz gráfica del HMI, una fase que se extendió durante aproximadamente un mes hasta alcanzar un funcionamiento completo y estable.

Finalizada la primera estación, se abordó la instalación del segundo PLC y de la estación unión, los cuales se integraron en el mismo proyecto de TIA Portal y se añadieron a la red PROFINET común, permitiendo así la comunicación entre todos los dispositivos del sistema. Del mismo modo, se conectaron las entradas y salidas de la estación de unión, y se comenzó con la programación de su lógica de control. Transcurrido otro mes de trabajo, ambas estaciones estaban totalmente automatizadas y operativas. En esta fase también se implementó la comunicación entre los dos PLCs, permitiendo así una coordinación eficiente que dio lugar a un proceso automatizado conjunto.

Finalmente, una vez completada la automatización de las estaciones, se procedió a la programación del robot colaborativo UR5e. Se desarrolló una secuencia de paletizado de cartones de leche, manipulados mediante una pinza hidráulica capaz de apilar hasta cuatro capas. Una vez definida la secuencia de trabajo del cobot, este fue incorporado a la red PROFINET del sistema, permitiendo su comunicación directa con los PLCs. Se implementó el intercambio de señales de control entre el UR5e y los controladores, integrando al robot dentro del mismo ciclo de operación automatizada. Esta última fase requirió aproximadamente un mes adicional.


