\chapter{Plataforma de desarrollo}
\label{cap:capitulo3}

\begin{flushright}
\begin{minipage}[]{10cm}
\emph{Cualquier cosa que pueda ser automatizada, será automatizada..}\\
\end{minipage}\\

Robert Cannon, \textit{Título}\\
\end{flushright}

\vspace{1cm}

En este capítulo se detallan los recursos hardware y software empleados para llevar a cabo el desarrollo de la aplicación. A nivel hardware, se han utilizado dos maquetas didácticas de automatización industrial, cada una equipada con un PLC, junto con una interfaz HMI y un brazo robótico colaborativo de la empresa Universal Robots (UR). En cuanto al software, se han utilizado distintos entornos de desarrollo específicos para cada dispositivo: el software TIA Portal para la programación de los PLCs y la configuración de la HMI, y la plataforma URSim para la simulación y control del brazo robótico. Además, se han empleado librerías propias de los fabricantes, sistemas operativos compatibles con las herramientas utilizadas y conexiones mediante protocolos estándar de comunicación industrial como Profinet. Aparte de la programación individual de cada componente del sistema, se ha llevado a cabo el montaje completo y el conexionado de todos los elementos desde su estado inicial de fábrica.

\section{Estación distribución}

\section{Estación unión}

\section{PLCs Siemens 1200s}

\section{HMI}

\section{Comunicaciones}

\section{Brazo UR5}





