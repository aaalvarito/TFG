\cleardoublepage

\chapter*{Resumen\markboth{Resumen}{Resumen}}

La automatización industrial y la robótica colaborativa han transformado los entornos productivos modernos, mejorando notablemente la eficiencia, reduciendo errores humanos y ofreciendo nuevas oportunidades en la formación técnica especializada. En este contexto, los sistemas didácticos permiten simular entornos reales para preparar adecuadamente a los futuros profesionales del sector industrial y tecnológico. \\

Este trabajo aborda la automatización de una línea de producción didáctica, compuesta por dos estaciones de Festo y un robot colaborativo UR5e. El objetivo principal ha sido integrar todos los elementos mediante comunicaciones industriales PROFINET y control mediante PLCs y HMI, reproduciendo un entorno industrial realista, funcional, escalable y aplicable en múltiples escenarios educativos y demostrativos. \\

Se ha utilizado TIA Portal para la programación de las estaciones, aplicando metodologías como Grafcet y la Guía GEMMA para lograr un resultado robusto, profesional y estructurado. Además, el robot UR5e ha sido programado para realizar tareas de paletizado e integrado en la red PROFINET, permitiendo una operación sincronizada y eficiente dentro del ciclo completo de producción simulado. \\

Las pruebas se realizaron en un entorno físico real, logrando una automatización estable y eficaz del sistema completo en diversas condiciones. Los resultados obtenidos validan su utilidad como herramienta didáctica, adaptable a diferentes prácticas formativas, accesible para estudiantes y con gran potencial para la formación en automatización y robótica industrial avanzada.
